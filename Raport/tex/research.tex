\chapter{Тестирование}

Проведем несколько тестов на правильность решения задачи оптимизации. Тестирование будем проводить следующим образом: будем брать множество точек, проверить правильность решения задачи минимизации которой интуитивно не вызывает особого труда. Иными словами, будем брать $m$ точек для задачи, которые или удовлетворяют функции, или находятся в непосредственной близости к множеству точек, удовлетворяющим значениям функции.
Искомые функции будут иметь следующий вид:
\begin{itemize}
	\item для линейной функции в N-мерном пространства в качестве примера рассмотрим $F(\vec{x}) = 1 + x_1 + x_2$;
	\item для полинома -- $P(x) = 1 + 3x + 3x^2 - x^3$;
	\item для кусочно-линейной функции - $f(x) = -1 + |x+1|-|x|+|x+1|$;
	\item для сглаживающего сплайна возьмем пример из учебника "Метод конечных элементов для решения скалярных и векторных задач" на странице 214.
\end{itemize}

Искомые функции изображены на рисунках \ref{fig:LineN} -- \ref{fig:Spline}.

\begin{figure}
	\centering
	\vspace*{0.7cm}
	\includegraphics[width=0.5\linewidth]{images/"Flat1".png}
	\caption{Линейная функция в 3-х мерном пространстве}
	\label{fig:LineN}
\end{figure}


\begin{figure}
	\centering
	\vspace*{0.7cm}
	\includegraphics[width=0.5\linewidth]{images/"Flat2".png}
	\caption{Линейная функция в 3-х мерном пространстве (вид сверху)}
	\label{fig:LineN1}
\end{figure}


\begin{figure}
	\centering
	\vspace*{0.7cm}
	\includegraphics[width=0.5\linewidth]{images/"Flat3".png}
	\caption{Линейная функция в 3-х мерном пространстве (вид сбоку)}
	\label{fig:LineN2}
\end{figure}


\begin{figure}
	\centering
	\vspace*{0.7cm}
	\includegraphics[width=0.5\linewidth]{images/"Polynomial".png}
	\caption{Полиномиальная функция}
	\label{fig:Polynomial}
\end{figure}

\begin{figure}
\centering
\vspace*{0.7cm}
\includegraphics[width=0.6\linewidth]{images/PW.png}
\caption{Кусочно-линейная функция}
\label{fig:PW}
\end{figure}

\begin{figure}
\centering
\vspace*{0.7cm}
\includegraphics[width=0.5\linewidth]{images/"Spline".png}
\caption{Сглаживающий сплайн}
\label{fig:Spline}
\end{figure}

Коэффициенты решения для сглаживающего сплайна выглядит следующим образом: $q = \left[2.5, -0.0117, 2.48, -0.0673, 0.0488, -3.35\right]$.

Тестовые входные данные с координатами точек хранятся в директории \text{/SciDevOOP/bin/Resources}. Содержимое этих файлов представлено в таблице \ref{tab:testData}.

\begin{table}
	\caption{Содержимое файлов с входными данными}
	\centering
	\small
	\begin{tabularx}{1.0\textwidth}{| >{\raggedright\arraybackslash}X | >{\raggedright\arraybackslash}X | >{\raggedright\arraybackslash}X | >{\raggedright\arraybackslash}X |}
		\hline
		\centering{Линейная 3-х мерная}  & \centering{Полином} & \centering{Кусочно-постоянная} & \centering{Сплайн} \tabularnewline \hline    
			\text{7}\newline
			\text{0 0 1} \newline
			\text{1 0 2} \newline
			\text{0 1 2} \newline
			\text{1 1 3} \newline
			\text{0,5 0 1,5} \newline
			\text{0 0,5 1,5} \newline
			\text{0,5 0,5 2} \newline  & 
			\text{15} \newline
			\text{-2,5 27,875} \newline
			\text{-2 15} \newline
			\text{-1,5 6,625} \newline
			\text{-1 2} \newline
			\text{-0,5 0,375} \newline
			\text{0 1} \newline
			\text{0,5 3,125} \newline
			\text{1 6} \newline
			\text{1,5 8,875} \newline
			\text{2 11} \newline
			\text{2,5 11,625} \newline
			\text{3 10} \newline
			\text{3,5 5,375} \newline
			\text{4 -3} \newline
			\text{4,5 -15,87} \newline & 
			\text{9} \newline
			\text{-2 1} \newline
			\text{-1,5 0,5} \newline
			\text{-1 0} \newline
			\text{-0,5 0,5} \newline
			\text{0 1} \newline
			\text{0,5 0,5} \newline
			\text{1 0} \newline
			\text{1,5 0,5} \newline
			\text{2 1} & 
			\text{11} \newline
			\text{0,0 2,5} \newline
			\text{0,5 2,5} \newline
			\text{1,1 2,5} \newline
			\text{1,5 2,5} \newline
			\text{2,0 2,5} \newline
			\text{2,5 2,5} \newline
			\text{3,0 2,5} \newline
			\text{4,0 2,3} \newline
			\text{4,5 1,2} \newline
			\text{4,8 0,8} \newline
			\text{5,0 0,0} \newline \tabularnewline \hline    
	\end{tabularx}
	\label{tab:testData}
\end{table}

\section{Тестирование метода имитации отжига}


\begin{table}
	\caption{Тестирование 3-х мерной линейной функции $P(\vec{x}) = 1 + x_1 + x_2$.}
	\centering
	\small
	\begin{tabularx}{1.0\textwidth}{| >{\raggedright\arraybackslash}X | >{\raggedright\arraybackslash}X | >{\raggedright\arraybackslash}X |}
		\hline
		\centering{Начальные параметры}  & \centering{Функционал} & \centering{Результат} \tabularnewline \hline    
		
		\multirow{4}{*}{\centering{(0.5; 0.5; 0.5)}} & $L_1$ & \centering{0.00000000E+000; 0.00000000E+000; 0.00000000E+000} \tabularnewline \cline{2-3}
	
		 & $L_2$ & \centering{0.00000000E+000; 0.00000000E+000; 0.00000000E+000} \tabularnewline \cline{2-3}
	
		 & $L_{\inf}$ & \centering{0.00000000E+000; 0.00000000E+000; 0.00000000E+000} \tabularnewline \cline{2-3}
	
		 & Интеграл & \centering{0.00000000E+000; 0.00000000E+000; 0.00000000E+000} \tabularnewline \hline
	\end{tabularx}
	\label{tab:testLineN1}
\end{table}

\begin{table}
	\caption{Тестирование кусочно-линейной функции $f(x) = -1 + |x + 1| + |x| + |x - 1|$.}
	\centering
	\small
	\begin{tabularx}{1.0\textwidth}{| >{\raggedright\arraybackslash}X | >{\raggedright\arraybackslash}X | >{\raggedright\arraybackslash}X |}
		\hline
		\centering{Начальные параметры}  & \centering{Функционал} & \centering{Результат} \tabularnewline \hline    
		
		\multirow{4}{*}{\centering{(0.5; 0.5; 0.5)}} & $L_1$ & \centering{0.00000000E+000; 0.00000000E+000; 0.00000000E+000} \tabularnewline \cline{2-3}
		
		& $L_2$ & \centering{0.00000000E+000; 0.00000000E+000; 0.00000000E+000} \tabularnewline \cline{2-3}
		
		& $L_{\inf}$ & \centering{0.00000000E+000; 0.00000000E+000; 0.00000000E+000} \tabularnewline \cline{2-3}
		
		& Интеграл & \centering{0.00000000E+000; 0.00000000E+000; 0.00000000E+000} \tabularnewline \hline
	\end{tabularx}
	\label{tab:testPW1}
\end{table}

\begin{table}
	\caption{Тестирование полиномиальной функции $P(x) = 1 + 3x + 3x^2 - x^3$.}
	\centering
	\small
	\begin{tabularx}{1.0\textwidth}{| >{\raggedright\arraybackslash}X | >{\raggedright\arraybackslash}X | >{\raggedright\arraybackslash}X |}
		\hline
		\centering{Начальные параметры}  & \centering{Функционал} & \centering{Результат} \tabularnewline \hline    
		
		\multirow{4}{*}{\centering{(0.5; 0.5; 0.5)}} & $L_1$ & \centering{0.00000000E+000; 0.00000000E+000; 0.00000000E+000} \tabularnewline \cline{2-3}
		
		& $L_2$ & \centering{0.00000000E+000; 0.00000000E+000; 0.00000000E+000} \tabularnewline \cline{2-3}
		
		& $L_{\inf}$ & \centering{0.00000000E+000; 0.00000000E+000; 0.00000000E+000} \tabularnewline \cline{2-3}
		
		& Интеграл & \centering{0.00000000E+000; 0.00000000E+000; 0.00000000E+000} \tabularnewline \hline
	\end{tabularx}
	\label{tab:testPolynomial1}
\end{table}

\begin{table}
	\caption{Тестирование сплайна.}
	\centering
	\small
	\begin{tabularx}{1.0\textwidth}{| >{\raggedright\arraybackslash}X | >{\raggedright\arraybackslash}X | >{\raggedright\arraybackslash}X |}
		\hline
		\centering{Начальные параметры}  & \centering{Функционал} & \centering{Результат} \tabularnewline \hline    
		
		\multirow{4}{*}{\centering{(0.5; 0.5; 0.5)}} & $L_1$ & \centering{0.00000000E+000; 0.00000000E+000; 0.00000000E+000} \tabularnewline \cline{2-3}
		
		& $L_2$ & \centering{0.00000000E+000; 0.00000000E+000; 0.00000000E+000} \tabularnewline \cline{2-3}
		
		& $L_{\inf}$ & \centering{0.00000000E+000; 0.00000000E+000; 0.00000000E+000} \tabularnewline \cline{2-3}
		
		& Интеграл & \centering{0.00000000E+000; 0.00000000E+000; 0.00000000E+000} \tabularnewline \hline
	\end{tabularx}
	\label{tab:testSpline1}
\end{table}

\section{Тестирование метода сопряжённых градиентов}

\begin{table}
	\caption{Тестирование 3-х мерной линейной функции $P(\vec{x}) = 1 + x_1 + x_2$.}
	\centering
	\small
	\begin{tabularx}{1.0\textwidth}{| >{\raggedright\arraybackslash}X | >{\raggedright\arraybackslash}X | >{\raggedright\arraybackslash}X |}
		\hline
		\centering{Начальные параметры}  & \centering{Функционал} & \centering{Результат} \tabularnewline \hline    
		
		\multirow{4}{*}{\centering{(0.5; 0.5; 0.5)}} & $L_1$ & \centering{0.00000000E+000; 0.00000000E+000; 0.00000000E+000} \tabularnewline \cline{2-3}
		
		& $L_2$ & \centering{0.00000000E+000; 0.00000000E+000; 0.00000000E+000} \tabularnewline \cline{2-3}
		
		& $L_{\inf}$ & \centering{0.00000000E+000; 0.00000000E+000; 0.00000000E+000} \tabularnewline \cline{2-3}
		
		& Интеграл & \centering{0.00000000E+000; 0.00000000E+000; 0.00000000E+000} \tabularnewline \hline
	\end{tabularx}
	\label{tab:testLineN2}
\end{table}

\begin{table}
	\caption{Тестирование кусочно-линейной функции $f(x) = -1 + |x + 1| + |x| + |x - 1|$.}
	\centering
	\small
	\begin{tabularx}{1.0\textwidth}{| >{\raggedright\arraybackslash}X | >{\raggedright\arraybackslash}X | >{\raggedright\arraybackslash}X |}
		\hline
		\centering{Начальные параметры}  & \centering{Функционал} & \centering{Результат} \tabularnewline \hline    
		
		\multirow{4}{*}{\centering{(0.5; 0.5; 0.5)}} & $L_1$ & \centering{0.00000000E+000; 0.00000000E+000; 0.00000000E+000} \tabularnewline \cline{2-3}
		
		& $L_2$ & \centering{0.00000000E+000; 0.00000000E+000; 0.00000000E+000} \tabularnewline \cline{2-3}
		
		& $L_{\inf}$ & \centering{0.00000000E+000; 0.00000000E+000; 0.00000000E+000} \tabularnewline \cline{2-3}
		
		& Интеграл & \centering{0.00000000E+000; 0.00000000E+000; 0.00000000E+000} \tabularnewline \hline
	\end{tabularx}
	\label{tab:testPW2}
\end{table}

\begin{table}
	\caption{Тестирование полиномиальной функции $P(x) = 1 + 3x + 3x^2 - x^3$.}
	\centering
	\small
	\begin{tabularx}{1.0\textwidth}{| >{\raggedright\arraybackslash}X | >{\raggedright\arraybackslash}X | >{\raggedright\arraybackslash}X |}
		\hline
		\centering{Начальные параметры}  & \centering{Функционал} & \centering{Результат} \tabularnewline \hline    
		
		\multirow{4}{*}{\centering{(0.5; 0.5; 0.5)}} & $L_1$ & \centering{0.00000000E+000; 0.00000000E+000; 0.00000000E+000} \tabularnewline \cline{2-3}
		
		& $L_2$ & \centering{0.00000000E+000; 0.00000000E+000; 0.00000000E+000} \tabularnewline \cline{2-3}
		
		& $L_{\inf}$ & \centering{0.00000000E+000; 0.00000000E+000; 0.00000000E+000} \tabularnewline \cline{2-3}
		
		& Интеграл & \centering{0.00000000E+000; 0.00000000E+000; 0.00000000E+000} \tabularnewline \hline
	\end{tabularx}
	\label{tab:testPolynomial2}
\end{table}

\begin{table}
	\caption{Тестирование сплайна.}
	\centering
	\small
	\begin{tabularx}{1.0\textwidth}{| >{\raggedright\arraybackslash}X | >{\raggedright\arraybackslash}X | >{\raggedright\arraybackslash}X |}
		\hline
		\centering{Начальные параметры}  & \centering{Функционал} & \centering{Результат} \tabularnewline \hline    
		
		\multirow{4}{*}{\centering{(0.5; 0.5; 0.5)}} & $L_1$ & \centering{0.00000000E+000; 0.00000000E+000; 0.00000000E+000} \tabularnewline \cline{2-3}
		
		& $L_2$ & \centering{0.00000000E+000; 0.00000000E+000; 0.00000000E+000} \tabularnewline \cline{2-3}
		
		& $L_{\inf}$ & \centering{0.00000000E+000; 0.00000000E+000; 0.00000000E+000} \tabularnewline \cline{2-3}
		
		& Интеграл & \centering{0.00000000E+000; 0.00000000E+000; 0.00000000E+000} \tabularnewline \hline
	\end{tabularx}
	\label{tab:testSpline2}
\end{table}


\section{Тестирование метода Левенберга — Марквардта}

\begin{table}
	\caption{Тестирование 3-х мерной линейной функции $P(\vec{x}) = 1 + x_1 + x_2$.}
	\centering
	\small
	\begin{tabularx}{1.0\textwidth}{| >{\raggedright\arraybackslash}X | >{\raggedright\arraybackslash}X | >{\raggedright\arraybackslash}X |}
		\hline
		\centering{Начальные параметры}  & \centering{Функционал} & \centering{Результат} \tabularnewline \hline    
		
		\multirow{4}{*}{\centering{(0.5; 0.5; 0.5)}} & $L_1$ & \centering{0.00000000E+000; 0.00000000E+000; 0.00000000E+000} \tabularnewline \cline{2-3}
		
		& $L_2$ & \centering{0.00000000E+000; 0.00000000E+000; 0.00000000E+000} \tabularnewline \cline{2-3}
		
		& $L_{\inf}$ & \centering{0.00000000E+000; 0.00000000E+000; 0.00000000E+000} \tabularnewline \cline{2-3}
		
		& Интеграл & \centering{0.00000000E+000; 0.00000000E+000; 0.00000000E+000} \tabularnewline \hline
	\end{tabularx}
	\label{tab:testLineN3}
\end{table}

\begin{table}
	\caption{Тестирование кусочно-линейной функции $f(x) = -1 + |x + 1| + |x| + |x - 1|$.}
	\centering
	\small
	\begin{tabularx}{1.0\textwidth}{| >{\raggedright\arraybackslash}X | >{\raggedright\arraybackslash}X | >{\raggedright\arraybackslash}X |}
		\hline
		\centering{Начальные параметры}  & \centering{Функционал} & \centering{Результат} \tabularnewline \hline    
		
		\multirow{4}{*}{\centering{(0.5; 0.5; 0.5)}} & $L_1$ & \centering{0.00000000E+000; 0.00000000E+000; 0.00000000E+000} \tabularnewline \cline{2-3}
		
		& $L_2$ & \centering{0.00000000E+000; 0.00000000E+000; 0.00000000E+000} \tabularnewline \cline{2-3}
		
		& $L_{\inf}$ & \centering{0.00000000E+000; 0.00000000E+000; 0.00000000E+000} \tabularnewline \cline{2-3}
		
		& Интеграл & \centering{0.00000000E+000; 0.00000000E+000; 0.00000000E+000} \tabularnewline \hline
	\end{tabularx}
	\label{tab:testPW3}
\end{table}

\begin{table}
	\caption{Тестирование полиномиальной функции $P(x) = 1 + 3x + 3x^2 - x^3$.}
	\centering
	\small
	\begin{tabularx}{1.0\textwidth}{| >{\raggedright\arraybackslash}X | >{\raggedright\arraybackslash}X | >{\raggedright\arraybackslash}X |}
		\hline
		\centering{Начальные параметры}  & \centering{Функционал} & \centering{Результат} \tabularnewline \hline    
		
		\multirow{4}{*}{\centering{(0.5; 0.5; 0.5)}} & $L_1$ & \centering{0.00000000E+000; 0.00000000E+000; 0.00000000E+000} \tabularnewline \cline{2-3}
		
		& $L_2$ & \centering{0.00000000E+000; 0.00000000E+000; 0.00000000E+000} \tabularnewline \cline{2-3}
		
		& $L_{\inf}$ & \centering{0.00000000E+000; 0.00000000E+000; 0.00000000E+000} \tabularnewline \cline{2-3}
		
		& Интеграл & \centering{0.00000000E+000; 0.00000000E+000; 0.00000000E+000} \tabularnewline \hline
	\end{tabularx}
	\label{tab:testPolynomial3}
\end{table}

\begin{table}
	\caption{Тестирование сплайна.}
	\centering
	\small
	\begin{tabularx}{1.0\textwidth}{| >{\raggedright\arraybackslash}X | >{\raggedright\arraybackslash}X | >{\raggedright\arraybackslash}X |}
		\hline
		\centering{Начальные параметры}  & \centering{Функционал} & \centering{Результат} \tabularnewline \hline    
		
		\multirow{4}{*}{\centering{(0.5; 0.5; 0.5)}} & $L_1$ & \centering{0.00000000E+000; 0.00000000E+000; 0.00000000E+000} \tabularnewline \cline{2-3}
		
		& $L_2$ & \centering{0.00000000E+000; 0.00000000E+000; 0.00000000E+000} \tabularnewline \cline{2-3}
		
		& $L_{\inf}$ & \centering{0.00000000E+000; 0.00000000E+000; 0.00000000E+000} \tabularnewline \cline{2-3}
		
		& Интеграл & \centering{0.00000000E+000; 0.00000000E+000; 0.00000000E+000} \tabularnewline \hline
	\end{tabularx}
	\label{tab:testSpline3}
\end{table}