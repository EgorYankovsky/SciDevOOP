\chapter{Практическая часть}

Надо ли?

\section{Функционалы}

\section{Функции}

При реализации функций в основе лежали следующие идеи их представления.

\begin{itemize}
		\item \textsl{Линейная функция в n-мерном пространстве} будет искаться в виде $f(\overline{x}) = c_0 + c_1 \cdot x_1 + c_2 \cdot x_2 + ... + c_n \cdot x_n$. Вектор входных параметров для данной функции выглядит следующим образом: [$c_0, c_1, ..., c_n$].
		\item \textsl{Полином n-й степени в одномерном пространстве} будет искаться в виде $P(x) = c_0 + c_1 \cdot x + c_2 \cdot x^2 + ... + c_n \cdot x^n$. Вектор входных параметров для данной функции выглядит следующим образом: [$c_0, c_1, ..., c_n$].
		\item \textsl{Кусочно-линейную функции} представим в виде системы уравнений $a \cdot x + b \cdot y = c, \forall x \in [x_i, x_{i + 1}]$. Тогда вектор входных параметров будет иметь вид: $[x0, x1, ... x_{n - 1}, a_0, a_1, ... a_n, b_0, b_1, ... b_n, c_0, c_1, ... c_n]$, где $[x_0, x_1, ... x_{n-1}]$ - координаты точек разлома, $[a_0, a_1, ... a_n]$ - коэффициенты при $x$, $[b_0, b_1, ... b_n]$ - коэффициенты при $y$, $[c_0, c_1, ... c_n]$ - свободные коэффициенты.
		\item \textsl{Сглаживающий сплайн} вида $S(x) = \sum_{i = 0}^{2n} q_i \cdot \psi_i(x)$ на эрмитовых базисных функциях. Вектор входных параметров определяется видом: $[q_0, ..., q_{2n}, x_0, ..., x_n]$.
	\end{itemize}
\section{Методы оптимизации}

В качестве методов оптимизации были реализованы следующие алгоритмы:

\begin{itemize}
	\item универсальный -- \textsl{Алгоритм имитации отжига};
	\item требующий \texttt{IDifferentiableFunctional} -- \textsl{Метод сопряжённых градиентов};
	\item требующий \texttt{ILeastSquaresFunctional} -- \textsl{Алгоритм Левенберга — Марквардта}.
\end{itemize}
