\chapter{Рекомендации по использованию программы}

Входная точка запуска программы находится в файле \texttt{Program.cs}. Содержимое файла выглядит следующим образом:

\begin{minted}[linenos=false, numbersep=0pt, frame=none]{csharp}
partial class Program
{
	static void Main(string[] args)
	{
		// 1. Choose IOptimizator: MinimizerLevenbergMarquardt, MinimizerMCG, MinimizerSimulatedAnnealing.
		IOptimizator optimizer = new MinimizerLevenbergMarquardt();
		// 2. Choose IParametricFunction: LineFunctionN, PiecewiseLinearFunction, Polynomial, SplineFunction.
		var fun = new PiecewiseLinearFunction();
		// 3. Read points.
		/*
		* Reading support console and TXT input.
		*  - to use console input call as: Read()
		*  - to use txt input call as: Read("file_name.txt")
		*      ! basicly, it reads files from \\SciDevOOP\\bin\\Resources\\ folder
		*      ! also you can set here a full path of EXISTING! file.
		*/
		var points = Read("inputPW.txt");
		// 4. Initialize vector of parameters.
		var initial = new Vector
		{
			0.7,
			0.4,
			0.1, 0.1, 0.2,
			-1.0, 0.0, 1.0
		};
		// 4a. If necessary - initialize vector of minimal or maximal parameters.
		var minimal = new Vector
		{
			0.0, -2.0, 0.0, -1.0, -1.0, -4.0,
			0.0, 2.0, 5.0
		};
		var maximal = new Vector
		{
			4.0, 2.0, 4.47, 1.78, 2.88, 0.35,
			0.0, 2.0, 5.0
		};
		// 5. Choose IFunctional: IntegrationNorm, L1Norm, L2Norm, LInfNorm.
		var functional = new L2Norm()
		{
			points = points
		};
		// 6. Solve minimization problem.
		var res = optimizer.Minimize(functional, fun, initial/*, minimal, maximal*/);
		// 7. Write solution.
		/*
		*  Writing support console and TXT output.
		*  - to use console output use as: Write(res)
		*  - to use txt output use as: Read(res, "file_name.txt")
		*      ! basicly, it uses \\SciDevOOP\\bin\\Resources\\ as output folder
		*      ! also you can set here a random folder.
		*/
		Write(res);
	}
}
\end{minted}

Работа всей программы состоит из 7 частей.

\begin{enumerate}
	\item \textbf{Выбор метода оптимизации.} Пользователю предлагается выбрать один из реализованных методов: Метод симуляции отжига, Метод сопряжённых градиентов и алгоритм Левенберга — Марквардта.
	\begin{itemize}
		\item По умолчанию, Метод симуляции отжига использует в качестве закона изменения температуры закон Коши (\texttt{BasketFiring}), а правило перехода используется метод последовательного улучшения (\texttt{BasketFiring}). Функционал программы позволяет пользователю выбрать и другие реализованные законы изменения температуры и правила перехода, описанные в теоретической части.
		\item При работе с алгоритмами, требующими дифференцируемый функционал, как параметр решения задачи оптимизации, пользователю также предлагается выбор из двух методов ограничения допустимых параметров: методы штрафных и барьерных функций. По умолчанию ставятся штрафные.
		\item При решении задачи минимизации функционала методом Левенберга — Марквардта, предлагается указать решатель СЛАУ. Допустимые реализации: метод Гаусса и ?LU-разложение?. По умолчанию стоит метод Гаусса.
	\end{itemize}
	\item \textbf{Выбор функции.} Пользователю предлагается выбрать один из реализованных функций: линейная функция в n-мерном пространстве, кусочно-линейная функция, полином n-ой степени и сглаживающий сплайн на эрмитовых базисных функциях.
	\item \textbf{Чтение точек.} Поддерживается чтение из консоли, а также *.txt файлов. Для чтения с консоли в функцию \texttt{Read} не нужно передавать никаких параметров. Для чтения данных с файлов необходимо указать полный путь расположения файла. Также определённый набор тестовых входных файлов находится в директории \texttt{./SciDevOOP/bin/Resources/} относительно файла решения \texttt{SciDevOOP.sln}. При необходимости чтения данных из этого файла пользователь может указать название файла, без полного пути к нему.
	\item \textbf{Инициализация вектора начальных параметров}, а также при необходимости минимальных и максимальных. Пользователю предлагается собственноручно заполнить три вектора значений инициализирующих, минимальных и максимальных параметров. Главное, чтобы количество элементов в них совпадало. Если размер вектора инициализирующих параметров не совпадает с размером векторов минимальных или максимальных параметров, то соответствующий вектор не будет учитываться при решении задачи. 
	\item \textbf{Выбор функционала для минимизации.} Пользователю на выбор предлагается один из реализованных функционалов: $L_1$-норма, $L_2$-норма, $L_{\inf}$-норма или же норма, как интеграл, рассчитываемый численно. (Хочется несколько методов, например Гаусс, Прямоугольники и вот это все).
	\item \textbf{Решение задачи минимизации.} На данном этапе происходит решение задачи минимизации функционала по заданному набору точек. В случае если не будут соблюдены требования метода оптимизации к функционалу или искомой функции, будет вызвано соответствующее исключение и решатель вернет пустой вектор.
	\item \textbf{Вывод точек.} Поддерживается запись в консоль, а также в *.txt файлы. Для записи в консоль функция \texttt{Write} должна принимать один параметр -- вектор решения из п.6 -- \texttt{sln}. Для записи данных в файл необходимо указать вторым аргументом полный путь расположения файла. Если же пользователь пропишет только название файла, то вектор решения будет записан в соответствующий файл, находящийся в директории \texttt{./SciDevOOP/bin/Resources/} относительно файла решения \texttt{SciDevOOP.sln}.
\end{enumerate}

Стоит учесть, что в решении для некоторых функций, использующих сетку как параметр (\texttt{PiecewiseLinearFunction}, \texttt{SplineFunction}), сама сетка не является искомым параметром. Подразумевается, что точки $[x_0, x_1, ..., x_n]$ определены пользователем верно.